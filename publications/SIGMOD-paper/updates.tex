\section{Updating Column Imprints}\label{sec:updates}

Column imprints are designed to support read intensive database applications.
In such scenarios, updates are a relatively rare event, and when they occur,
are performed in batches. The most common type of updates is appending new
rows of data to the end of a table. Column imprints can easily cope with such
updates. However, we can not exclude from our study updates that change an
arbitrary value of a column, or insert/delete a row in the middle of a table.

\subsection{Data Append}

During data appends, any index that is based on bit vectors and bit-binning
techniques has to perform two operations. The first one is to readjust, if
necessary, the borders of the bins. Such a readjustment should be avoided since
it calls for a complete rebuild of the index. For column imprints, this is very
rare, since \it{i)} the first and last bins are used for overflow values, and
\it{ii)} the bins were determined by sampling the active domain of the column.
Any new data appended, need to have dramatically different value
distribution to render the initial binning inefficient. The second operation
is to update the bit vectors. For bitmap indexes this is a costly
operation, since all bit vectors have to be updated, even those that are not
mapping the new values~\cite{CGF07}. For column imprints this is not necessary.
The imprint vectors are horizontally compressed, thus data appends simply cause
new imprint vectors to be appended to the end of the existing ones, without the
need of accessing any of the previous imprint vectors.

\subsection{Imprints and Delta Structures}

In place updates are never performed in columnar databases because of the
prohibitive cost they entail. Instead, a delta structure is used that keeps
track of the updates, and merges them at query time. A delta structure can
be as simple as two tables with insertions and deletions that need to be
union-ed and difference-ed, respectively, with the base table, or it can be
a more complex structure, such as positional update trees~\cite{HZN+10}.

Column imprints can cope with inter-column operations, such as unions and
differences, by first applying them to the cacheline dictionaries, such that a
candidate list of qualifying cachelines is created for both operands. The
details of inter-column operations are out of the scope of this paper, and are
left to be presented in the future. Nevertheless, even without such a
functionality, column imprints can be used to access the base table to create
a candidate list of qualifying cachelines. The underlying delta structure may
then hold in addition the cacheline counter where an update has been performed
in order to merge to the final result.

Moreover, imprints can produce false positives, thus a deletion can be ignored
by the corresponding imprint vector. An insertion however, will call for
additional bits to be set to the imprint corresponding to the affected
cachelines. Such an approach will eventually saturate the imprint index. In
these cases, it is not uncommon to disregard the entire secondary index and
rebuild it during the next query scan. The overhead for rebuilding an imprint
index during a regular scan in minimal, such that it will go undetected by the
user.
