\section{Related Work}\label{sec:background}
Column imprints can be viewed as a new member of the big family of bitmapped
based indexes. Bitmapped indexes have become the prime solution to deal with
the dimensionality curse of traditional index structures such as B-trees and
R-trees. Their contribution to
speed up processing has been credited to Patrick O'Neil through the work on the
Model 204 Data Management System~\cite{ON87,OG95}. Since then, database engines
include bitmapped indexes for both fast access over persistent data and as
intermediate storage scheme during query processing, e.g. Sybase IQ,
Postgresql, IBM DB2, Oracle. Besides traditional bitmaps, Bloom
filters~\cite{B70} have been used to decide if a record can be found in a
relation, and thus postponing bringing the data into memory. However, Bloom
filters are not suited for range queries, the target of column imprints.

Bitmap indexing relies on three orthogonal techniques~\cite{WOS06}:
binning, encoding and compression. Binning concerns the decision of how many
bit vectors to define. For low cardinality domains, a single bit vector for
each distinct value is used. High cardinality domains are dealt
with each bit vector representing a set of values. The common strategy is to
use a data value histogram to derive a number of equally sized bins. Although
binning reduces the number of bit vectors to manage, it also requires a post
analysis over the underlying table to filter out false positives during query
evaluation. Column imprints use similar binning techniques.

Since each record turns on a single bit in one bit vector of the index only,
the bitmaps become amendable to compression. Variations of run-length encoded
compression have been proposed. The state-of-the art approach is the
Word-Aligned Hybrid (WAH)~\cite{WOS02,WOS08} storage scheme. WAH forms the
heart of the open-source package
FastBit\footnote{http://crd-legacy.lbl.gov/\~{ }kewu/fastbit/}, which
is a mature collection of independent tools and a C++ library for indexing file
repositories. Consequently, column imprints use another variation of
run-length encoded but for identical cacheline mappings instead of consecutive
equal values.

Bitmap indexing has been used in large scientific database applications, such
as high-energy physics, network traffic analysis, lasers, and earth
sciences.
However, deployment of bitmap indexing over large-scale scientific databases is
disputed. \cite{SW07} claims that based on information theoretic constructs,
the length of a compressed interval encoded bitmap it too large when
high cardinality attributes are indexed. The storage size may become orders of
magnitude larger than the base data. Instead, a multi-level indexing scheme is
proposed to aid in the design of an optimal binning strategy. They extend the
work on bit binning~\cite{GS09,WOS04}. Alternatively, the data distribution in
combination with query workload can be used to refine the binning
strategy~\cite{Kou00,CI99}.

With the advent of multi-core and gpu processors it becomes attractive to
exploit data parallel algorithms to speed up processing.  Bit vectors carry the
nice property of being small enough to fit in the limited gpu memory, while
most bit operations nicely fall in the SIMD algorithm space. Promising results
have been reported in~\cite{GWB09}. Similar, re-engineering the algorithms to
work well in a flash storage architecture have shown significant
improvements~\cite{WMC10}.
