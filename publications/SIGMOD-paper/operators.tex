\section{Imprints Query Evaluation}\label{sec:operators}

In this section we present the algorithms for evaluating range queries over the
column imprints index. Given a range query
$\mathcal{Q}=[\it{low},\it{high}]$, all values \it{v} in column \it{col} that
satisfy $\it{low}\leq\it{v}\leq\it{high}$ need to be located. Since our
setting is a columnar database, it suffices to return the \it{id} list
of the qualifying values \it{v}.

Evaluating range queries over column imprints is a straightforward procedure.
The first step is to create an empty bit-vector and set the bits that
correspond to the bins that are included in the range of query $\mathcal{Q}$.
There might be more than one bits set, since the query range can span
multiple bins. The query bit-vector is then checked against the imprint
vectors, and if bitwise intersection indicates common bits set for both the
query and the imprint vector, the corresponding cacheline is accessed for
further processing. However, if all bits set correspond to bins that are fully
included in the query range $[\it{low},\it{high}]$ the cacheline need not
be checked at all. Otherwise, the algorithm examines all values in the
cacheline to weed out false positives. Finally, because of our compression
schema, some administrative overhead to keep the cachelines and the imprint
vectors aligned is needed.

\begin{algorithm}[t]
\caption{Evaluate range queries over the column imprints index:
\tt{query()}}
\label{alg:select}
\small{
%\tt{
\bf{Input:}  imprints index structure \it{imp}, column \it{col},
query $\mathcal{Q}=[\it{low},\it{high}]$\\
\bf{Output:} array \it{res} of \it{id}s
\begin{tabbing}
char $\it{vpc}$; \` /* constant values per cacheline */\\
ulong $\it{i\_cnt}=0$;\` /* imprints count */\\
ulong $\it{cache\_cnt}=0$;\` /* cacheline count */\\
ulong $\it{id}=0$;\` /* \it{id}s counter */\\
ulong $\it{*res}$;\` /* large enough array to hold the result */\\
(\it{mask}, \it{innermask}) = \tt{make\_masks}(\it{imp},[\it{low},\it{high}]);\\
\bf{for}\=\ $\it{i}=0\rightarrow \it{d\_cnt}-1$ \bf{do}
\` /* iterate over the cacheline dictionary */\\
\>\bf{if}\=\ $(\it{imp.cd}[\it{i}]\it{.repeat}\equiv 0)$ \bf{then}
\` /* if \it{repeat} is not set */\\
\>\> \bf{for}\=\ $\it{j}=\it{i\_cnt}\rightarrow \it{i\_cnt}+
\it{imp.cd}[\it{i}]\it{.cnt}-1$ \bf{do}\\
\>\>\>\bf{if}\=\ $(\it{imp.imprints}[\it{j}]\& \it{mask})$ \bf{then}
\` /* if imprint vector matches mask */\\
\>\>\>\>\bf{if}\=\ $((\it{imp.imprints}[\it{j}]\&\ \tilde{\ }\it{innermask})
\equiv 0)$ \bf{then}\\
\>\>\>\>\>\bf{for}\=\ $\it{id}=\it{cache\_cnt}\times\it{vpc}\rightarrow
(\it{cache\_cnt}\times(\it{vpc}+1))-1$ \bf{do}\\
\>\>\>\>\>\>$\it{res}=\it{res}\leftarrow\it{id}$;
\` /* add \it{id} to the result set \it{res} */\\
\>\>\>\>\>\bf{end for}\\
\>\>\>\>\bf{else}\` /* need to check for false-positives */\\
\>\>\>\>\>\bf{for}\=\ $\it{id}=\it{cache\_cnt}\times\it{vpc}\rightarrow
(\it{cache\_cnt}\times(\it{vpc}+1))-1$ \bf{do}\\
\>\>\>\>\>\>\bf{if}\=\ $(\it{col}[id]<\it{high}\wedge\it{col}[id]
\geq\it{low})$ \bf{then}\\
\>\>\>\>\>\>\>$\it{res}=\it{res}\leftarrow\it{id}$;
\` /* add \it{id} to the result set \it{res} */\\
\>\>\>\>\>\>\bf{end if}\\
\>\>\>\>\>\bf{end for}\\
\>\>\>\>\bf{end if}\\
\>\>\>\bf{end if}\\
\>\>\>$\it{cache\_cnt}\ += 1$;\` /* increase cache count by 1*/\\
\>\>\bf{end for}\\
\>\>$\it{i\_cnt}\ +=\it{imp.cd}[\it{i}]\it{.cnt}$; \` /* increase imprint count */\\
\>\bf{else}\`  /* \it{repeat} is set */\\
\>\>\bf{if}\=\ $(\it{imp.imprints}[\it{i\_cnt}]\& \it{mask})$ \bf{then}
\` /* if imprint vector match mask */\\
\>\>\>\bf{if}\=\ $((\it{imp.imprints}[\it{f\_count}]\&\ \tilde{\ }\it{innermask})
\equiv 0)$ \bf{then}\\
\>\>\>\>\bf{for}\=\ $\it{id}=\it{cache\_cnt}$\=$\times\it{vpc}\rightarrow$\\
\>\>\>\>\>\>$(\it{cache\_cnt}\times\it{vpc})+\it{vpc}\times\it{imp.cd}[\it{i}]\it{.cnt}-1$ \bf{do}\\
\>\>\>\>\>$\it{res}=\it{res}\leftarrow\it{id}$;
\` /* add \it{id} to the result set \it{res} */\\
\>\>\>\>\bf{end for}\\
\>\>\>\bf{else}\` /* need to check for false-positives */\\
\>\>\>\>\bf{for}\=\ $\it{id}=\it{cache\_cnt}$\=$\times\it{vpc}\rightarrow$\\
\>\>\>\>\>\>$(\it{cache\_cnt}\times\it{vpc})+\it{vpc}\times\it{imp.cd}[\it{i}]\it{.cnt}-1$ \bf{do}\\
\>\>\>\>\>\bf{if}\=\ $(\it{col}[id]<\it{high}\wedge\it{col}[id]
\geq\it{low})$ \bf{then}\\
\>\>\>\>\>\>$\it{res}=\it{res}\leftarrow\it{id}$;
\` /* add \it{id} to the result set \it{res} */\\
\>\>\>\>\>\bf{end if}\\
\>\>\>\>\bf{end for}\\
\>\>\>\bf{endif}\\
\>\>\bf{end if}\\
\>\>$\it{i\_cnt}\ += 1$; \` /* increase imprints count by 1 */\\
\>\>$\it{cache\_cnt}\ += \it{imp.cd}[\it{i}]\it{.cnt}$;
\` /* increase cache count */\\
\>\bf{end if}\\
\bf{end for}
\end{tabbing}
}%}
\vspace{-10pt}
\end{algorithm}

Algorithm~\ref{alg:select} presents the details for evaluating a range query
using imprints. The constant \it{vpc} is set equal to the number of values that
fit in a cacheline. This is needed to align \it{id}s with the cachelines. In
addition, counters \it{i\_cnt} and \it{cache\_cnt} are maintained to align
imprints and cachelines, respectively. Next, two bit-vectors are produced,
namely \it{mask} and \it{innermask}. The \it{mask} is a bit-vector that sets
all bits that fall into the range $[\it{low},\it{high}]$. The \it{innermask} is
a bit-vector with only the bits that fall entirely inside the query range set.
More precisely, if a bin range contains one of the borders of the query range,
the corresponding bit is not set. Therefore, if an imprint vector
has only the bits from the \it{innermask} set, then all values in the
corresponding cacheline fall into the query range and no further check for
false-positives is needed. The algorithm runs by iterating over all
entries in the cacheline dictionary. If the \it{repeat} flag is not set, then
the next \it{cnt} imprint vectors correspond to \it{cnt} distinct cachelines.
For any of these imprints, if there is at least one bit set in the same
position as the ones in the \it{mask} bit-vector, the cacheline contains
values that satisfy the query range. If in addition, there are no bits set
different than the bits of the \it{innermask}, then all the values of the
cacheline satisfy the query. In any other case, we need to check each value
of the cacheline individually. For all qualifying values, the corresponding
\it{id}s are materialized in the result array. If however the \it{repeat} flag
is set, then by checking only one imprint vector we can determine if the next
\it{cnt} cachelines contain values that fall into the range of the query. As
before, an extra check with the \it{innermask} bit-vector may result in
avoiding the check of each individual value for false-positives.

Algorithm~\ref{alg:select} returns a materialized list of the \it{id}s that
satisfy the range query. This list is then passed to the next operator of the
query evaluation engine. However, it might be the case that a user's query
contains many predicates for more than one attribute of the same relation. In
this case, the \tt{query()} procedure of Algorithm~\ref{alg:select} is invoked
multiple times, one for each attribute, with possible different
$[\it{low},\it{high}]$ values. The most expensive part of
Algorithm~\ref{alg:select} is the check for false-positives and the
materialization of the \it{id}s. But in the case of multiple range queries over
many columns of the same table, both of these expensive operations can be
postponed. This technique is known in the literature as late materialization.
To achieve this, instead of producing the materialized \it{id} lists,
Algorithm~\ref{alg:select} has to return the list of the qualifying cachelines.
After every range query is evaluated over the respective columns, the lists of
cachelines are merge-joined, resulting in a smaller set of qualifying
\it{id}s. This is based on the general expectation that the combination of many
range queries will increase the selectivity of the final result set. After the
merge-join, the qualifying \it{id}s that were common to all cachelines can be
checked for false-positives. Note that the alternative indexing schemes used
in the evaluation of Section~\ref{sec:evaluation} have been coded with the same
rigidity.
